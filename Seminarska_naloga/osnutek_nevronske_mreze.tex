
\documentclass[a4paper,12pt]{article}
\usepackage[utf8]{inputenc}
\usepackage[slovene]{babel}
\usepackage{amsmath, amssymb}
\usepackage{graphicx}
\usepackage{hyperref}
\usepackage{listings}
\usepackage{xcolor}
\usepackage{geometry}
\geometry{margin=2.5cm}

\title{Učenje z nevronskimi mrežami v Pythonu}
\author{}
\date{\today}

\definecolor{codegray}{rgb}{0.5,0.5,0.5}
\definecolor{codepurple}{rgb}{0.58,0,0.82}
\definecolor{backcolour}{rgb}{0.95,0.95,0.92}

\lstdefinestyle{mystyle}{
    backgroundcolor=\color{backcolour},   
    commentstyle=\color{codegray},
    keywordstyle=\color{blue},
    numberstyle=\tiny\color{codegray},
    stringstyle=\color{codepurple},
    basicstyle=\ttfamily\footnotesize,
    breaklines=true,
    captionpos=b,
    keepspaces=true,
    numbers=left,
    numbersep=5pt,
    showspaces=false,
    showstringspaces=false,
    showtabs=false,
    tabsize=2
}

\lstset{style=mystyle}

\begin{document}

\maketitle

\section*{Povzetek}
V tej nalogi sem raziskoval uporabo umetnih nevronskih mrež za prepoznavo vzorcev. Uporabil sem programski jezik Python ter knjižnico Keras za izdelavo enostavne mreže, ki razvršča ročno napisane številke iz baze podatkov MNIST. Preverjal sem, kako število slojev, nevronov in parametrov vpliva na točnost rezultatov. Cilj je bil ustvariti osnovno razvrščevalno nevronsko mrežo ter razumeti njen način delovanja.

\textbf{Ključne besede:} umetna inteligenca, strojno učenje, nevronska mreža, Python, MNIST

\section*{Uvod}
Umetne nevronske mreže so osnova sodobnega strojnega učenja. Uporabljajo se v mnogih področjih, kot so prepoznavanje govora, slik, avtonomna vozila in medicinska diagnostika. Glavna ideja naloge je bila razumeti, kako zgraditi preprosto nevronsko mrežo in jo naučiti prepoznavanja številk. Pri tem sem se osredotočil predvsem na razumevanje osnovnega delovanja mreže in preizkušanje vpliva različnih nastavitev na delovanje.

\section*{Opis naloge}

\subsection*{Načrtovanje}
Za izvedbo naloge sem se odločil uporabiti Python zaradi njegove enostavne sintakse in bogate podpore za strojno učenje. Cilj je bil ustvariti nevronsko mrežo, ki zna prepoznati številke iz podatkovne baze MNIST, ki vsebuje 60.000 učnih in 10.000 testnih slik ročno napisanih številk.

\subsection*{Uporabljena programska oprema}
\begin{itemize}
  \item Python 3.10
  \item Keras kot vmesnik za TensorFlow
  \item Matplotlib za prikaz slik
  \item NumPy za obdelavo podatkov
\end{itemize}

\subsection*{Zgradba mreže}
Mrežo sem sestavil iz:
\begin{itemize}
  \item vhodnega sloja: 784 vhodov (28x28 pikslov)
  \item skritega sloja: 128 nevronov z aktivacijsko funkcijo ReLU
  \item izhodnega sloja: 10 nevronov z aktivacijsko funkcijo softmax (za 10 različnih številk)
\end{itemize}

\begin{lstlisting}[language=Python, caption=Zgradba modela v Keras]
model = keras.Sequential([
    keras.layers.Flatten(input_shape=(28, 28)),
    keras.layers.Dense(128, activation='relu'),
    keras.layers.Dense(10, activation='softmax')
])
\end{lstlisting}

Model sem treniral z uporabo metode \texttt{model.fit()} in testiral njegovo natančnost z \texttt{model.evaluate()}.

\subsection*{Rezultati}
Mreža je dosegla natančnost okoli 97 \% na testnih podatkih. Preizkusil sem tudi več različnih velikosti skritih slojev in ugotovil, da prevelika mreža ne izboljša natančnosti bistveno, lahko pa povzroči prekomerno učenje.

\section*{Zaključek}
S projektom sem pridobil osnovno razumevanje nevronskih mrež in njihove uporabe v praksi. Spoznal sem pomen predobdelave podatkov, izbire arhitekture mreže in različnih hiperparametrov. V prihodnje bi želel poskusiti z bolj zapletenimi arhitekturami, kot so konvolucijske nevronske mreže (CNN), ki so bolj primerne za delo s slikami.

\section*{Priloge}
\begin{itemize}
  \item Izsek programske kode
  \item Graf učenja (natančnost skozi epohe)
  \item Primeri prepoznanih številk
\end{itemize}

\end{document}
