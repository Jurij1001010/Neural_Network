
\documentclass[12pt,a4paper]{article}
\usepackage[utf8]{inputenc}
\usepackage[slovene]{babel}
\usepackage{amsmath,amsfonts,amssymb}
\usepackage{graphicx}
\usepackage{hyperref}
\usepackage{listings}
\usepackage{color}
\usepackage{caption}
\usepackage{float}
\usepackage{geometry}
\geometry{margin=2.5cm}

\definecolor{codegray}{rgb}{0.5,0.5,0.5}
\definecolor{codeblue}{rgb}{0.25,0.5,0.75}
\definecolor{backcolour}{rgb}{0.95,0.95,0.92}

\lstdefinestyle{mystyle}{
    backgroundcolor=\color{backcolour},   
    commentstyle=\color{codegray},
    keywordstyle=\color{codeblue},
    numberstyle=\tiny\color{codegray},
    stringstyle=\color{codeblue},
    basicstyle=\ttfamily\footnotesize,
    breaklines=true,
    captionpos=b,
    keepspaces=true,
    numbers=left,
    numbersep=5pt,
    showspaces=false,
    showstringspaces=false,
    showtabs=false,
    tabsize=2
}
\lstset{style=mystyle}

\title{Lastna implementacija nevronske mreže v Javi brez zunanjih knjižnic}
\author{Ime Priimek \\ Fakulteta za računalništvo in informatiko}
\date{April 2025}

\begin{document}

\maketitle
\tableofcontents
\newpage

\section{Uvod}
Cilj seminarske naloge je predstaviti koncept umetnih nevronskih mrež skozi lastno implementacijo v programskem jeziku Java brez uporabe zunanjih knjižnic. S tem pristopom pridobimo globlje razumevanje osnov delovanja in učenja nevronskih mrež. V praktičnem delu pokažemo delovanje na dveh enostavnih klasifikacijskih primerih.

\section{Teoretično ozadje}
\subsection{Umetni nevron}
Umetni nevron je matematični model biološkega nevrona. Vsak nevron prejme več vhodov $x_1, x_2, ..., x_n$, ki jih uteži $w_1, w_2, ..., w_n$ utežijo, ter doda pristranskost (bias) $b$. Izhod se nato izračuna kot:
\[
z = \sum_{i=1}^n w_i x_i + b
\]
\[
a = \phi(z)
\]
kjer je $\phi$ aktivacijska funkcija.

\subsection{Aktivacijske funkcije}
Uporabljene aktivacijske funkcije:
\begin{itemize}
    \item \textbf{Sigmoid:} $\phi(z) = \frac{1}{1 + e^{-z}}$
    \item \textbf{ReLU:} $\phi(z) = \max(0, z)$
\end{itemize}

\subsection{Feedforward arhitektura}
Gre za usmerjeno mrežo brez povratnih povezav. Informacija se širi samo naprej od vhodnega sloja proti izhodnemu.

\subsection{Učenje in gradientni spust}
Izguba se izračuna z npr. srednjo kvadratno napako:
\[
L = \frac{1}{n} \sum (y_{res} - y_{true})^2
\]
Gradientni spust se uporablja za prilagajanje uteži v smeri zmanjšanja napake.

\section{Implementacija v Javi}
\subsection{Struktura programa}
Implementacija je razdeljena na naslednje razrede:
\begin{itemize}
    \item \texttt{Neuron} – predstavlja en nevron
    \item \texttt{Layer} – sloj nevronske mreže
    \item \texttt{NeuralNetwork} – glavni razred mreže
\end{itemize}

\subsection{Primer ključnega dela kode}
\begin{lstlisting}[language=Java, caption=Izračun izhoda nevrona]
public double activate(double[] inputs) {
    double sum = bias;
    for (int i = 0; i < inputs.length; i++) {
        sum += weights[i] * inputs[i];
    }
    return sigmoid(sum);
}
\end{lstlisting}

\section{Praktični primeri}
\subsection{Primer 1: Razločevanje nad/pod premico $y = 0$}
Točke $(x, y)$ so razvrščene kot:
\begin{itemize}
    \item Razred 1: $y > 0$
    \item Razred 0: $y \leq 0$
\end{itemize}

\subsection{Primer 2: Znotraj/zunaj kroga}
Središče kroga $(0,0)$ in polmer $r = 1$. Razvrstitev:
\begin{itemize}
    \item Razred 1: $x^2 + y^2 \leq 1$
    \item Razred 0: $x^2 + y^2 > 1$
\end{itemize}

\section{Rezultati}
\begin{figure}[H]
\centering
\includegraphics[width=0.6\textwidth]{primer1_graf.png}
\caption{Klasifikacija nad/pod premico}
\end{figure}

\begin{figure}[H]
\centering
\includegraphics[width=0.6\textwidth]{primer2_krog.png}
\caption{Klasifikacija znotraj/zunaj kroga}
\end{figure}

\section{Zaključek}
Seminarska naloga je pokazala, da je mogoče osnovno nevronsko mrežo uspešno implementirati tudi brez knjižnic. Praktični primeri klasifikacije kažejo sposobnost mreže, da se nauči ločevanja vzorcev.

\section{Literatura}
\begin{itemize}
    \item S. Haykin, \textit{Neural Networks and Learning Machines}
    \item Michael Nielsen, \textit{Neural Networks and Deep Learning}
    \item Razni spletni viri in dokumentacije Jave
\end{itemize}

\end{document}
